% How to run: tectonic cv.tex

\documentclass[10pt,a4paper]{article}

% Identifying information
\newcommand{\Title}{Academic CV}
\newcommand{\FirstName}{Bartosz}
\newcommand{\LastName}{Bednarczyk}
\newcommand{\Initials}{BBe}
\newcommand{\MyName}{\FirstName\ \LastName}
\newcommand{\Me}{\underline{\LastName, \Initials}}  % For citations
\newcommand{\Email}{bartosz.bednarczyk@cs.uni.wroc.pl}
\newcommand{\PersonalWebsite}{bartoszjanbednarczyk.github.io}
\newcommand{\LabWebsite}{bartoszjanbednarczyk.github.io}
\newcommand{\ORCID}{0000-0002-8267-7554}
\newcommand{\GitHubProfile}{bartoszjanbednarczyk}


% Load packages
%%%%%%%%%%%%%%%%%%%%%%%%%%%%%%%%%%%%%%%%%%%%%%%%%%%%%%%%%%%%%%%%%%%%%%%%%%%%%%%

% Full Unicode support for non-ASCII characters
\usepackage[utf8]{inputenc}
\usepackage[english]{babel}
\usepackage[TU]{fontenc}

% Set main fonts
\usepackage[sfdefault]{atkinson}
\usepackage[ttdefault]{sourcecodepro}

% Icon fonts
\usepackage{fontawesome5}
\usepackage{academicons}

% Disable hyphenation
\usepackage[none]{hyphenat}

% Control the font size
\usepackage{anyfontsize}

% For fancy and multipage tables
\usepackage{tabularx}
\usepackage{ltablex}

% For new environments
\usepackage{environ}

% Manage dates and times
\usepackage{datetime}

% Set the page margins
\usepackage{geometry}

% To get the total page numbers (\pageref{LastPage})
\usepackage{lastpage}

% Control spacing in enumerates
\usepackage{enumitem}

% Use custom colors
\usepackage[usenames,dvipsnames]{xcolor}

% Configure section titles
\usepackage{titlesec}

% Fancy header configuration
\usepackage{fancyhdr}

% Control PDF metadata and links
\usepackage[colorlinks=true]{hyperref}


% Template configuration
%%%%%%%%%%%%%%%%%%%%%%%%%%%%%%%%%%%%%%%%%%%%%%%%%%%%%%%%%%%%%%%%%%%%%%%%%%%%%%%

\geometry{%
  margin=12.5mm,
  headsep=0mm,
  headheight=0mm,
  footskip=5mm,
  includehead=true,
  includefoot=true
}

% Custom colors
\definecolor{mediumgray}{gray}{0.5}
\definecolor{lightgray}{gray}{0.9}
\definecolor{mediumblue}{HTML}{2060c2}
\definecolor{lightblue}{HTML}{a0c3ff}

% No indentation
\setlength\parindent{0cm}

% Increase the line spacing
\renewcommand{\baselinestretch}{1.1}
% and the spacing between rows in tables
\renewcommand{\arraystretch}{1.25}

% Remove space between items in itemize and enumerate
\setlist{nosep}

% Set the spacing and format of sections
\titleformat{\section}
  {\normalfont\Large\mdseries} % format
  {} % label
  {0pt} % separation (left separation for hang)
  {} % text before title
  [\titlerule] % text after title
\titlespacing*{\section}
  {0pt} % left pad
  {0.1cm} % before
  {0cm} % after

% Disable number of sections. Use this instead of "section*" so that the sections still
% appear as PDF bookmarks. Otherwise, would have to add the table of contents entries
% manually.
\makeatletter
\renewcommand{\@seccntformat}[1]{}
\makeatother

% Define a new environment to place all CV entries in a 2-column table.
% Left column are the dates, right column the entries.
\newcommand{\TablePad}{\vspace{-0.2cm}}
\NewEnviron{EntriesTableDuration}{
\TablePad
\begin{tabularx}{\textwidth}{@{}p{0.135\textwidth}@{\hspace{0.02\textwidth}}p{0.845\textwidth}@{}}
  \BODY
\end{tabularx}
\TablePad
}
\NewEnviron{EntriesTableYear}{
\TablePad
\begin{tabularx}{\textwidth}{@{}p{0.08\textwidth}@{\hspace{0.01\textwidth}}p{0.91\textwidth}@{}}
  \BODY
\end{tabularx}
\TablePad
}

% Macros to set the year and duration on the left column
\newcommand{\Duration}[2]{\fontsize{10pt}{0}\selectfont \texttt{#1-#2}}
\newcommand{\Year}[1]{\fontsize{10pt}{0}\selectfont \texttt{#1}}
\newcommand{\Ongoing}{on}
\newcommand{\Future}{future}

% Macros to add links and mark publications
\newcommand{\DOI}[1]{DOI: \href{https://doi.org/#1}{#1}}
\newcommand{\Website}[1]{\href{https://#1}{#1}}
\newcommand{\Preprint}[1]{Preprint: \href{https://doi.org/#1}{#1}}
\newcommand{\GitHub}[1]{GitHub: \href{https://github.com/#1}{#1}}

% Define command to insert month name and year as date
\newdateformat{monthyear}{\monthname[\THEMONTH], \THEYEAR}

% Configure a fancy footer
\newcommand{\Separator}{\hspace{3pt}|\hspace{3pt}}
\newcommand{\FooterFont}{\footnotesize\color{mediumgray}}
\pagestyle{fancy}
\fancyhf{}
\lfoot{%
  \FooterFont{}
  \MyName{}
  \Separator{}
  \Title{}
}
\rfoot{%
  \FooterFont{}
  Last updated: \monthyear\today{}
  \Separator{}
  \thepage\space of\space \pageref*{LastPage}
}
\renewcommand{\headrulewidth}{0pt}
\renewcommand{\footrulewidth}{1pt}
\preto{\footrule}{\color{lightgray}}

% Metadata for the PDF output and control of hyperlinks
\hypersetup{
  colorlinks,
  allcolors=mediumblue,
  breaklinks=true,
  pdftitle={\Title{} - \MyName},
  pdfauthor={\MyName},
}


%%%%%%%%%%%%%%%%%%%%%%%%%%%%%%%%%%%%%%%%%%%%%%%%%%%%%%%%%%%%%%%%%%%%%%%%%%%%%%%
\begin{document}

\begin{minipage}[t]{0.5\textwidth}
  {\fontsize{20pt}{0}\selectfont\MyName}
\end{minipage}
\begin{minipage}[t]{0.5\textwidth}
  \begin{flushright}
    \Title{}
  \end{flushright}
\end{minipage}
\\[-0.1cm]
\textcolor{lightgray}{\rule{\textwidth}{3pt}}
\begin{minipage}[t]{0.5\textwidth}
  ORCID: \href{https://orcid.org/\ORCID}{\ORCID}
  \\
  Website: \Website{\PersonalWebsite}
  \\
  Email: \href{mailto:\Email}{\Email}
\end{minipage}
\begin{minipage}[t]{0.5\textwidth}
  \begin{flushright}
  Institute of Computer Science, University of Wrocław\\
  Fryderyka Joliot-Curie 15,\\
  Room 324\\
  50-383 Wrocław, Poland
  % \\
  % Computational Logic Group, ICCL@TU Dresden,
  % \\
  % Nöthnitzer Str. 46, 01187 Dresden, Germany 
  \end{flushright}
\end{minipage}
\vspace{0.3cm}

% %%%%%%%%%%%%%%%%%%%%%%%%%%%%%%%%%%%%%%%%%%%%%%%%%%%%%%%%%%%%%%%%%%%%%%%%%%%%%%%

\section{Education}

\begin{EntriesTableDuration}
  \Year{25.06.2024}  &
  \textbf{PhD in Computer Science (with distinction)}, Technische Universität Dresden, Germany. \newline
  Thesis: \emph{Database-Inspired Reasoning Problems in Description Logics With Path Expressions} \newline
  Supervisors: Sebastian Rudolph (TU Dresden) and Emanuel Kieroński (University of Wrocław) \newline
  Reviewers: Sebastian Rudolph (TU Dresden) and Magdalena Ortiz (TU Wien)
  \\
  \Year{01.10.2018} & \textbf{M1 Master Parisien de Recherche en Informatique}, École Normale Supérieure Paris-Saclay, France \newline
  Thesis: \emph{Assertion Languages with Modalities and Separating Connectives} \newline
  Supervisor: Stéphane Demri (LMF, CNRS \& ENS Paris-Saclay) \newline
  Grade: Magna cum laude
  \\
  \Year{15.02.2017} & \textbf{BSc in Computer Science}, University of Wrocław, Poland \newline
  Thesis: \emph{Satisfiability of the Two-Variable Fragment of FO with Counting Quantifiers over Finite Trees} \newline
  Supervisor: Witold Charatonik (University of Wrocław) \newline
\end{EntriesTableDuration}

%%%%%%%%%%%%%%%%%%%%%%%%%%%%%%%%%%%%%%%%%%%%%%%%%%%%%%%%%%%%%%%%%%%%%%%%%%%%%%%

\section{Positions}

\begin{EntriesTableDuration}
  \Duration{10.2024}{$\infty$}  &
  \textbf{Assistant Professor}, University of Wrocław, Poland
\end{EntriesTableDuration}

\begin{EntriesTableDuration}
  \Duration{04.'19}{09.'24}  &
  \textbf{Research Associate}, Technische Universität Dresden, Germany
\end{EntriesTableDuration}

%%%%%%%%%%%%%%%%%%%%%%%%%%%%%%%%%%%%%%%%%%%%%%%%%%%%%%%%%%%%%%%%%%%%%%%%%%%%%%%

\section{Participation in Research Grants}

\begin{EntriesTableDuration}
  \Duration{04.'19}{09.'24}  & 
  \textbf{Research Associate}: ERC Consolidator Grant DeciGUT (PI: Sebastian Rudolph)\newline
  \emph{Hosting institution}: TU Dresden, Germany
\end{EntriesTableDuration}

\begin{EntriesTableDuration}
  \Duration{10.'18}{09.'22}  & 
  \textbf{Principal Investigator}: Polish Min. of Science and H. Education ``Diamond Grant''  DI2017006447.\newline
  \emph{Hosting institution}: University of Wrocław, Poland\newline
  Realised \emph{with distinction}.
\end{EntriesTableDuration}

\begin{EntriesTableDuration}
  \Duration{03.'17}{06.'18}  & 
  \textbf{Student Assistant}: Polish National Science Centre Grant 2016/21/B/ST6/01444 (PI: E. Kieroński)\newline
  \emph{Hosting institution}: University of Wrocław, Poland
\end{EntriesTableDuration}

%%%%%%%%%%%%%%%%%%%%%%%%%%%%%%%%%%%%%%%%%%%%%%%%%%%%%%%%%%%%%%%%%%%%%%%%%%%%%%%

\section{Longer Research Stays}

\begin{EntriesTableYear}
  \Year{2021 (2mth)}  & 
  Research visit (Topic: Model Theory of Ordered Logics)\newline
  \emph{Hosted by}: Tampere University (Antti Kuusisto)
\end{EntriesTableYear}

\begin{EntriesTableYear}
  \Year{2018 (3mth)}  & 
  Research visits (Topic: Query Entailment in Description Logics)\newline
  \emph{Hosted by}: TU Dresden (Sebastian Rudolph)
\end{EntriesTableYear}

\begin{EntriesTableYear}
  \Year{2018 (3mth)}  & 
  Research Internship (Topic: Algebraic Characterisations of Tree Languages)\newline
  \emph{Hosted by}: University of Oxford (James Worrell \& Michaël Cadilhac)
\end{EntriesTableYear}

\begin{EntriesTableYear}
  \Year{2017 (3mth)}  & 
  Research Internship (Topic: Temporal Logics and Weighted Automata)\newline
  \emph{Hosted by}: IST Austria (Krishnendu Chatterjee)
\end{EntriesTableYear}

%%%%%%%%%%%%%%%%%%%%%%%%%%%%%%%%%%%%%%%%%%%%%%%%%%%%%%%%%%%%%%%%%%%%%%%%%%%%%%%

\section{Teaching}

\begin{EntriesTableYear}
  \Duration{'18}{$\infty$} & University of Wrocław\newline
  Logic in Computer Science: (TA) 2018--2024, (Supplementary Lecture/Repetytorium): 2020, 2021, 2024\newline
  Research Seminar in Logic and Databases: summer 2018/19, winter 2019/20\newline
  Finite and Algorithmic Model Theory: (Lecturer + TA) 2020, 2022, 2025, (Seminar) 2020 \newline
  Databases: (TA) 2019, 2025\newline
  Discrete Mathematics: (TA) 2024\newline
  Python: (TA) 2024
\end{EntriesTableYear}

\begin{EntriesTableYear}
  \Duration{'21}{'22} & TU Dresden\newline
  Finite and Algorithmic Model Theory: (Lecturer + TA) 2021, 2022
\end{EntriesTableYear}

\begin{EntriesTableYear}
  \Duration{'21}{'24} &  XIV Highschool of Wrocław\newline
  Computer Science Teacher. Classes related to the Matura Exam in Computer Science.
\end{EntriesTableYear}

  % \Duration{'21}{'24} &\\
  %  @ XIV Highschool of Wrocław \\


% %%%%%%%%%%%%%%%%%%%%%%%%%%%%%%%%%%%%%%%%%%%%%%%%%%%%%%%%%%%%%%%%%%%%%%%%%%%%%%%

\section{Awards}

\begin{EntriesTableDuration}
  \Year{2023}  &
  Best student paper at JELIA 2023\\
\end{EntriesTableDuration}

\begin{EntriesTableDuration}
  \Year{2021}  &
  Award for Outstanding Young Scientists\newline Given by the Polish Ministry of Science and Higher Education (194 040 PLN)\\
\end{EntriesTableDuration}

\begin{EntriesTableDuration}
  \Year{2021}  &
  Best student paper at JELIA 2021\\
\end{EntriesTableDuration}

\begin{EntriesTableDuration}
  \Year{2019}  &
  Hugo Steinhaus Scholarship for Mathematical Sciences\newline Award for outstanding PhD students from Wrocław (18.000 PLN)\\
\end{EntriesTableDuration}

\begin{EntriesTableDuration}
  \Year{2018}  &
  Diamond Grant by the Polish Ministry of Science and Higher Education\newline Prestigious funding for own four-year research project\\
\end{EntriesTableDuration}

%%%%%%%%%%%%%%%%%%%%%%%%%%%%%%%%%%%%%%%%%%%%%%%%%%%%%%%%%%%%%%%%%%%%%%%%%%%%%%%

\section{Professional service}

\begin{EntriesTableDuration}

  \Year{2025}  &
  (co)Organizer of  : Description Logic Workshop 2025, VLDB Summer School 2025\newline
  PC: AAAI 2025, $\ldots$\newline
  Reviewer (Conferences): CSL 2025, $\ldots$\\



  \Year{2024}  &
  Reviewer (Conferences): AIML 2024, PODS 2024\newline
  Reviewer (Journals): Logic Journal of the IGPL, LMCS, SN Computer Science, Artificial Intelligence\newline
  PC: AAAI 2024, DL 2024, IJCAI 2024, KR 2024\\


  \Year{2023}  &
  Reviewer: EUMAS 2023 \newline
  PC: AAAI 2023, KR 2023, JELIA 2023, IJCAI 2023, DL 2023, DPFO Workshop 2023, TIME 2023\\

  \Year{2022}  &
  Reviewer: CSL 2022, LMCS, IPL, JAIR, ToCL \newline
  PC: IJCAI 2022, KR 2022, DL 2022\\

  \Year{2021}  &
  Reviewer: DL 2021, Elsevier Artificial Intelligence, Fundamenta Informaticae \newline
  Senior PC: IJCAI 2021\\

  \Year{2020}  &
  Reviewer: KR 2020, CONCUR 2020, DL 2020\\  
\end{EntriesTableDuration}


% %%%%%%%%%%%%%%%%%%%%%%%%%%%%%%%%%%%%%%%%%%%%%%%%%%%%%%%%%%%%%%%%%%%%%%%%%%%%%%%

\section{Student Supervision (Contact me if you are interested!)}

\begin{EntriesTableDuration}

  \Year{2024}  &
  Sebastian Zięciak (MSc, Ongoing)\newline
  Finite Controllability for Forward Existential Rules\\

  \Year{2024}  &
  Mikołaj Swoboda (BSc, Ongoing)\newline
  Finite Model Property for Hybrid Graded Mu-Calculus\\

  \Year{2024}  &
  Benno Fünfstück (MSc)\newline
  A Constructive Proof of a Van Benthem Theorem for the
  Forward Guarded Fragment of First Order~Logic\\

  \Year{2023}  &
  Karol Ochman-Milarski (BSc)\newline
  Resolution for Forward Guarded Fragment\\

  \Year{2023}  &
  Julita Osman, Aleksandra Stępniewska, Nikola Wrona (BEng) \newline
  E-learning platform supporting studying for the Matura exam in Computer Science\\

  \Year{2022}  &
  Mateusz Urbańczyk (MSc) \newline
  Categorical semantics for model comparison games for description logics\\

  \Year{2022}  &
  Johannes Tantow (Großer Beleg) \newline
  Interpolation for the Two-Variable Guarded Fragment with Counting\\

  \Year{2021}  &
  Oskar Fiuk (BSc) \newline
  Presburger Tree Automata With Applications to Logics With Expressive Counting\\

  \Year{2021}  &
  Sebastian Zięciak (BSc) \newline
  On Several Equivalent Characterisations of the Variety R\\

  \Year{2021}  &
  Maja Orłowska and Anna Pacanowska  (BSc) \newline
  Statistical constructions in decidable fragments of First-Order Logic\\

  \Year{2021}  &
  Martyna Siejba (MSc) \newline
  The complexity of the satisfiability problem for Modal Logics with Data over Heaps\\
    
\end{EntriesTableDuration}

% %%%%%%%%%%%%%%%%%%%%%%%%%%%%%%%%%%%%%%%%%%%%%%%%%%%%%%%%%%%%%%%%%%%%%%%%%%%%%%%

\section{Peer-Reviewed Conference Papers}

\begin{EntriesTableYear}

  \Year{2024}  &
  \textbf{Data Complexity in Expressive Description Logics With Path Expressions}.
  \newline
  IJCAI 2024
  \newline
 Bartosz Bednarczyk
  \\

\Year{2023}  &
  \textbf{Beyond ALCreg: Exploring Non-Regular Extensions of PDL with Description Logics Features}.
  \newline
  JELIA 2023 \textbf{(Best Student Paper Award)}
  \newline
 Bartosz Bednarczyk
  \\


\Year{2023}  &
  \textbf{On the Limits of Decision: the Adjacent Fragment of First-Order Logic}.
  \newline
  ICALP 2023
  \newline
 Bartosz Bednarczyk, Daumantas Kojelis, Ian Pratt-Hartmann 
  \\

\Year{2022}  &
  \textbf{Finite Entailment of Local Queries in the Z Family of Description Logics}.
  \newline
  AAAI 2022
  \newline
  Bartosz Bednarczyk, Emanuel Kieroński 
  \\

\Year{2022}  &
  \textbf{The Price of Selfishness: Conjunctive Query Entailment for ALCSelf Is 2EXPTIME-Hard}.
  \newline
  AAAI 2022
  \newline
  Bartosz Bednarczyk, Sebastian Rudolph
  \\

\Year{2022}  &
  \textbf{Towards a Model Theory of Ordered Logics: Expressivity and Interpolation}.
  \newline
  MFCS 2022
  \newline
  Bartosz Bednarczyk, Reijo Jaakkola
  \\

\Year{2022}  &
  \textbf{Presburger Büchi Tree Automata with Applications to Logics with Expressive Counting}.
  \newline
  WOLLIC 2022
  \newline
  Bartosz Bednarczyk, Oskar Fiuk 
  \\

\Year{2021}  &
  \textbf{"Most of" leads to undecidability: Failure of adding frequencies to LTL}.
  \newline
  FOSSACS 2021
  \newline
  Bartosz Bednarczyk, Jakub Michaliszyn 
  \\

\Year{2021}  &
  \textbf{On Classical Decidable Logics Extended with Percentage Quantifiers and Arithmetics}.
  \newline
  FSTTCS 2021
  \newline
  Bartosz Bednarczyk, Maja Orlowska, Anna Pacanowska, Tony Tan
  \\

\Year{2021}  &
  \textbf{Exploiting Forwardness: Satisfiability and Query-Entailment in Forward Guarded Fragment}.
  \newline
  JELIA 2021 \textbf{(Best Student Paper Award)}
  \newline
  Bartosz Bednarczyk
  \\

\Year{2020}  &
  \textbf{Satisfiability and Query Answering in Description Logics with Global and Local Cardinality Constraints}.
  \newline
  ECAI 2020
  \newline
  Franz Baader, Bartosz Bednarczyk, Sebastian Rudolph
  \\

\Year{2020}  &
  \textbf{A Framework for Reasoning about Dynamic Axioms in Description Logics}.
  \newline
  IJCAI 2020
  \newline
 Bartosz Bednarczyk, Stéphane Demri, Alessio Mansutti 
  \\

\Year{2020}  &
  \textbf{All-Instances Oblivious Chase Termination is Undecidable for Single-Head Binary TGDs}.
  \newline
  IJCAI 2020
  \newline
  Bartosz Bednarczyk, Robert Ferens, Piotr Ostropolski-Nalewaja
  \\

\Year{2020}  &
  \textbf{Modal Logics with Composition on Finite Forests: Expressivity and Complexity}.
  \newline
  LICS 2020
  \newline
  Bartosz Bednarczyk, Stéphane Demri, Raul Fervari, Alessio Mansutti
  \\

\Year{2020}  &
  \textbf{A Note on C2 Interpreted over Finite Data-Words}.
  \newline
  TIME 2020
  \newline
  Bartosz Bednarczyk, Piotr Witkowski
  \\

\Year{2019}  &
  \textbf{Worst-Case Optimal Querying of Very Expressive Description Logics}\newline \textbf{with Path Expressions and Succinct Counting}.
  \newline
  IJCAI 2019
  \newline
  Bartosz Bednarczyk, Sebastian Rudolph
  \\

\Year{2019}  &
  \textbf{On the Complexity of Graded Modal Logics with Converse}.
  \newline
  JELIA 2019
  \newline
  Bartosz Bednarczyk, Emanuel Kieroński, Piotr Witkowski
  \\

\Year{2019}  &
  \textbf{Why Propositional Quantification Makes Modal Logics on Trees Robustly Hard?}
  \newline
  LICS 2019
  \newline
  Bartosz Bednarczyk, Stéphane Demri 
  \\

\Year{2017}  &
  \textbf{Extending Two-Variable Logic on Trees}.
  \newline
  CSL 2017
  \newline
    Bartosz Bednarczyk, Witold Charatonik, Emanuel Kieroński
  \\

\Year{2017}  &
  \textbf{Modulo Counting on Words and Trees}.
  \newline
  FSTTCS 2017
  \newline
    Bartosz Bednarczyk, Witold Charatonik 
  \\
\end{EntriesTableYear}


% %%%%%%%%%%%%%%%%%%%%%%%%%%%%%%%%%%%%%%%%%%%%%%%%%%%%%%%%%%%%%%%%%%%%%%%%%%%%%%%

\section{Journal Papers}

\begin{EntriesTableYear}
  \Year{2024}  &
    \textbf{About the Expressive Power and Complexity of Order-Invariance with Two Variables}.
    \newline
    Logical Methods in Computer Science (accepted with major revisions) 
    \newline
    Bartosz Bednarczyk, Julien Grange
    \\
  \end{EntriesTableYear}

\begin{EntriesTableYear}
  \Year{2024}  &
    \textbf{Exploring Non-Regular Extensions of Propositional Dynamic Logic with Description-Logics Features.
    }.
    \newline
    Logical Methods in Computer Science 
    \newline
    Bartosz Bednarczyk
    \\
  \end{EntriesTableYear}

\begin{EntriesTableYear}
\Year{2023}  &
  \textbf{How to Tell Easy from Hard: Complexity of Conjunctive Query Entailment in Extensions of ALC}.
  \newline
  Journal of Artificial Intelligence Research 
  \newline
  Bartosz Bednarczyk, Sebastian Rudolph
  \\
\end{EntriesTableYear}

\begin{EntriesTableYear}
\Year{2023}  &
  \textbf{On Composing Finite Forests with Modal Logics}.
  \newline
  ACM Transactions on Computational Logic
  \newline
  Bartosz Bednarczyk, Stéphane Demri, Raul Fervari, Alessio Mansutti
  \\
\end{EntriesTableYear}

\begin{EntriesTableYear}
\Year{2022}  &
  \textbf{Why Does Propositional Quantification Make Modal and Temporal Logics on Trees Robustly Hard?}
  \newline
  Logical Methods in Computer Science
  \newline
 Bartosz Bednarczyk, Stéphane Demri
  \\
\end{EntriesTableYear}

\begin{EntriesTableYear}
\Year{2021}  &
  \textbf{Completing the Picture: Complexity of Graded Modal Logics with Converse}.
  \newline
  Theory and Practice of Logic Programming \textbf{(invited paper)}
  \newline
 Bartosz Bednarczyk, Emanuel Kieroński, Piotr Witkowski
  \\
\end{EntriesTableYear}

\begin{EntriesTableYear}
\Year{2021}  &
  \textbf{Statistical EL is ExpTime-complete}.
  \newline
  Information Processing Letters
  \newline
 Bartosz Bednarczyk
  \\
\end{EntriesTableYear}

\begin{EntriesTableYear}
\Year{2020}  &
  \textbf{One-variable logic meets Presburger arithmetic}.
  \newline
  Theoretical Computer Science
  \newline
 Bartosz Bednarczyk
  \\
\end{EntriesTableYear}


% %%%%%%%%%%%%%%%%%%%%%%%%%%%%%%%%%%%%%%%%%%%%%%%%%%%%%%%%%%%%%%%%%%%%%%%%%%%%%%%

\section{Informal Workshop Papers}

\begin{EntriesTableYear}
  \Year{2024}  &
    \textbf{On the Limits of Decision: the Adjacent Fragment of First-Order Logic (Extended Abstract)}
    \newline
    Description Logics 2024
    \newline
    Bartosz Bednarczyk, Daumantas Kojelis and Ian Pratt-Hartmann  
    \\
  \end{EntriesTableYear}

  \begin{EntriesTableYear}
    \Year{2024}  &
      \textbf{Data Complexity in Expressive Description Logics With Path Expressions And Its Applications in Rooted Conjunctive Query Entailment (Extended Abstract)}
      \newline
      Description Logics 2024
      \newline
      Bartosz Bednarczyk
      \\
    \end{EntriesTableYear}

\begin{EntriesTableYear}
\Year{2022}  &
  \textbf{Comonadic Semantics for Description Logics Games}
  \newline
  Description Logics 2022
  \newline
  Bartosz Bednarczyk, Mateusz Urbanczyk
  \\
\end{EntriesTableYear}

\begin{EntriesTableYear}
\Year{2021}  &
  \textbf{The Price of Selfishness: Conjunctive Query Entailment for ALCSelf is 2ExpTime-hard (Extended Abstract)}
  \newline
  Description Logics 2021
  \newline
 Bartosz Bednarczyk, Sebastian Rudolph
  \\
\end{EntriesTableYear}

\begin{EntriesTableYear}
\Year{2021}  &
  \textbf{Finite-Controllability of Conjunctive Queries in the Z family of Description Logics (Extended Abstract)}
  \newline
  Description Logics 2021
  \newline
 Bartosz Bednarczyk, Emanuel Kieroński
  \\
\end{EntriesTableYear}

\begin{EntriesTableYear}
\Year{2019}  &
  \textbf{Worst-Case Optimal Querying of Very Expressive Description Logics with Path Expressions and Succinct Counting}
  \newline
  Description Logics 2019
  \newline
 Bartosz Bednarczyk, Sebastian Rudolph
  \\
\end{EntriesTableYear}

\begin{EntriesTableYear}
\Year{2019}  &
  \textbf{
Satisfiability Checking and Conjunctive Query Answering in Description Logics with Global and Local Cardinality Constraint}.
  \newline
  Description Logics 2019
  \newline
 Franz Baader, Bartosz Bednarczyk, Sebastian Rudolph
  \\
\end{EntriesTableYear}


\end{document}
